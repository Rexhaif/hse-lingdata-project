% Options for packages loaded elsewhere
\PassOptionsToPackage{unicode}{hyperref}
\PassOptionsToPackage{hyphens}{url}
%
\documentclass[
]{article}
\usepackage{amsmath,amssymb}
\usepackage{lmodern}
\usepackage{ifxetex,ifluatex}
\ifnum 0\ifxetex 1\fi\ifluatex 1\fi=0 % if pdftex
  \usepackage[T1]{fontenc}
  \usepackage[utf8]{inputenc}
  \usepackage{textcomp} % provide euro and other symbols
\else % if luatex or xetex
  \usepackage{unicode-math}
  \defaultfontfeatures{Scale=MatchLowercase}
  \defaultfontfeatures[\rmfamily]{Ligatures=TeX,Scale=1}
\fi
% Use upquote if available, for straight quotes in verbatim environments
\IfFileExists{upquote.sty}{\usepackage{upquote}}{}
\IfFileExists{microtype.sty}{% use microtype if available
  \usepackage[]{microtype}
  \UseMicrotypeSet[protrusion]{basicmath} % disable protrusion for tt fonts
}{}
\makeatletter
\@ifundefined{KOMAClassName}{% if non-KOMA class
  \IfFileExists{parskip.sty}{%
    \usepackage{parskip}
  }{% else
    \setlength{\parindent}{0pt}
    \setlength{\parskip}{6pt plus 2pt minus 1pt}}
}{% if KOMA class
  \KOMAoptions{parskip=half}}
\makeatother
\usepackage{xcolor}
\IfFileExists{xurl.sty}{\usepackage{xurl}}{} % add URL line breaks if available
\IfFileExists{bookmark.sty}{\usepackage{bookmark}}{\usepackage{hyperref}}
\hypersetup{
  pdftitle={Essay analysis},
  hidelinks,
  pdfcreator={LaTeX via pandoc}}
\urlstyle{same} % disable monospaced font for URLs
\usepackage[margin=1in]{geometry}
\usepackage{color}
\usepackage{fancyvrb}
\newcommand{\VerbBar}{|}
\newcommand{\VERB}{\Verb[commandchars=\\\{\}]}
\DefineVerbatimEnvironment{Highlighting}{Verbatim}{commandchars=\\\{\}}
% Add ',fontsize=\small' for more characters per line
\usepackage{framed}
\definecolor{shadecolor}{RGB}{248,248,248}
\newenvironment{Shaded}{\begin{snugshade}}{\end{snugshade}}
\newcommand{\AlertTok}[1]{\textcolor[rgb]{0.94,0.16,0.16}{#1}}
\newcommand{\AnnotationTok}[1]{\textcolor[rgb]{0.56,0.35,0.01}{\textbf{\textit{#1}}}}
\newcommand{\AttributeTok}[1]{\textcolor[rgb]{0.77,0.63,0.00}{#1}}
\newcommand{\BaseNTok}[1]{\textcolor[rgb]{0.00,0.00,0.81}{#1}}
\newcommand{\BuiltInTok}[1]{#1}
\newcommand{\CharTok}[1]{\textcolor[rgb]{0.31,0.60,0.02}{#1}}
\newcommand{\CommentTok}[1]{\textcolor[rgb]{0.56,0.35,0.01}{\textit{#1}}}
\newcommand{\CommentVarTok}[1]{\textcolor[rgb]{0.56,0.35,0.01}{\textbf{\textit{#1}}}}
\newcommand{\ConstantTok}[1]{\textcolor[rgb]{0.00,0.00,0.00}{#1}}
\newcommand{\ControlFlowTok}[1]{\textcolor[rgb]{0.13,0.29,0.53}{\textbf{#1}}}
\newcommand{\DataTypeTok}[1]{\textcolor[rgb]{0.13,0.29,0.53}{#1}}
\newcommand{\DecValTok}[1]{\textcolor[rgb]{0.00,0.00,0.81}{#1}}
\newcommand{\DocumentationTok}[1]{\textcolor[rgb]{0.56,0.35,0.01}{\textbf{\textit{#1}}}}
\newcommand{\ErrorTok}[1]{\textcolor[rgb]{0.64,0.00,0.00}{\textbf{#1}}}
\newcommand{\ExtensionTok}[1]{#1}
\newcommand{\FloatTok}[1]{\textcolor[rgb]{0.00,0.00,0.81}{#1}}
\newcommand{\FunctionTok}[1]{\textcolor[rgb]{0.00,0.00,0.00}{#1}}
\newcommand{\ImportTok}[1]{#1}
\newcommand{\InformationTok}[1]{\textcolor[rgb]{0.56,0.35,0.01}{\textbf{\textit{#1}}}}
\newcommand{\KeywordTok}[1]{\textcolor[rgb]{0.13,0.29,0.53}{\textbf{#1}}}
\newcommand{\NormalTok}[1]{#1}
\newcommand{\OperatorTok}[1]{\textcolor[rgb]{0.81,0.36,0.00}{\textbf{#1}}}
\newcommand{\OtherTok}[1]{\textcolor[rgb]{0.56,0.35,0.01}{#1}}
\newcommand{\PreprocessorTok}[1]{\textcolor[rgb]{0.56,0.35,0.01}{\textit{#1}}}
\newcommand{\RegionMarkerTok}[1]{#1}
\newcommand{\SpecialCharTok}[1]{\textcolor[rgb]{0.00,0.00,0.00}{#1}}
\newcommand{\SpecialStringTok}[1]{\textcolor[rgb]{0.31,0.60,0.02}{#1}}
\newcommand{\StringTok}[1]{\textcolor[rgb]{0.31,0.60,0.02}{#1}}
\newcommand{\VariableTok}[1]{\textcolor[rgb]{0.00,0.00,0.00}{#1}}
\newcommand{\VerbatimStringTok}[1]{\textcolor[rgb]{0.31,0.60,0.02}{#1}}
\newcommand{\WarningTok}[1]{\textcolor[rgb]{0.56,0.35,0.01}{\textbf{\textit{#1}}}}
\usepackage{graphicx}
\makeatletter
\def\maxwidth{\ifdim\Gin@nat@width>\linewidth\linewidth\else\Gin@nat@width\fi}
\def\maxheight{\ifdim\Gin@nat@height>\textheight\textheight\else\Gin@nat@height\fi}
\makeatother
% Scale images if necessary, so that they will not overflow the page
% margins by default, and it is still possible to overwrite the defaults
% using explicit options in \includegraphics[width, height, ...]{}
\setkeys{Gin}{width=\maxwidth,height=\maxheight,keepaspectratio}
% Set default figure placement to htbp
\makeatletter
\def\fps@figure{htbp}
\makeatother
\setlength{\emergencystretch}{3em} % prevent overfull lines
\providecommand{\tightlist}{%
  \setlength{\itemsep}{0pt}\setlength{\parskip}{0pt}}
\setcounter{secnumdepth}{-\maxdimen} % remove section numbering
\ifluatex
  \usepackage{selnolig}  % disable illegal ligatures
\fi

\title{Essay analysis}
\author{}
\date{\vspace{-2.5em}}

\begin{document}
\maketitle

Here we will perform analysis of LIWC data of essays, written by
different groups of people.

\begin{Shaded}
\begin{Highlighting}[]
\FunctionTok{install.packages}\NormalTok{(}\FunctionTok{c}\NormalTok{(}\StringTok{"tidyverse"}\NormalTok{, }\StringTok{"dplyr"}\NormalTok{, }\StringTok{"ggplot2"}\NormalTok{, }\StringTok{"ggpubr"}\NormalTok{, }\StringTok{"coin"}\NormalTok{, }\StringTok{"rstatix"}\NormalTok{))}
\end{Highlighting}
\end{Shaded}

\begin{verbatim}
## Installing packages into '/home/rstudio-user/R/x86_64-pc-linux-gnu-library/4.0'
## (as 'lib' is unspecified)
\end{verbatim}

\begin{Shaded}
\begin{Highlighting}[]
\FunctionTok{library}\NormalTok{(tidyverse)}
\end{Highlighting}
\end{Shaded}

\begin{verbatim}
## -- Attaching packages --------------------------------------- tidyverse 1.3.1 --
\end{verbatim}

\begin{verbatim}
## v ggplot2 3.3.4     v purrr   0.3.4
## v tibble  3.1.2     v dplyr   1.0.7
## v tidyr   1.1.3     v stringr 1.4.0
## v readr   1.4.0     v forcats 0.5.1
\end{verbatim}

\begin{verbatim}
## -- Conflicts ------------------------------------------ tidyverse_conflicts() --
## x dplyr::filter() masks stats::filter()
## x dplyr::lag()    masks stats::lag()
\end{verbatim}

\begin{Shaded}
\begin{Highlighting}[]
\FunctionTok{library}\NormalTok{(dplyr)}
\FunctionTok{library}\NormalTok{(ggplot2)}
\FunctionTok{library}\NormalTok{(ggpubr)}
\FunctionTok{library}\NormalTok{(coin)}
\end{Highlighting}
\end{Shaded}

\begin{verbatim}
## Loading required package: survival
\end{verbatim}

\begin{Shaded}
\begin{Highlighting}[]
\FunctionTok{library}\NormalTok{(rstatix)}
\end{Highlighting}
\end{Shaded}

\begin{verbatim}
## 
## Attaching package: 'rstatix'
\end{verbatim}

\begin{verbatim}
## The following objects are masked from 'package:coin':
## 
##     chisq_test, friedman_test, kruskal_test, sign_test, wilcox_test
\end{verbatim}

\begin{verbatim}
## The following object is masked from 'package:stats':
## 
##     filter
\end{verbatim}

First, let's load the data. Concatenated dataset is in
\texttt{all\_groups.csv}

\begin{Shaded}
\begin{Highlighting}[]
\NormalTok{data }\OtherTok{\textless{}{-}} \FunctionTok{read.csv}\NormalTok{(}\StringTok{"https://raw.githubusercontent.com/Rexhaif/hse{-}lingdata{-}project/main/all\_groups.csv"}\NormalTok{)}
\NormalTok{data[data }\SpecialCharTok{==} \DecValTok{0}\NormalTok{] }\OtherTok{\textless{}{-}} \ConstantTok{NA}
\FunctionTok{head}\NormalTok{(data)}
\end{Highlighting}
\end{Shaded}

\begin{verbatim}
##                         Filename Segment  WC   WPS Sixltr   Dic Функция
## 1   E2T12HF7DVgDyo2apqO3Qw_d.txt       1 128 11.64  33.59 84.38   47.66
## 2   E6HnYOg_LBTxiFKNQbsNSw_d.txt       1 142 14.20  38.73 77.46   42.96
## 3   Y-oL9e5ZdMN5DytW7XMqGg_d.txt       1 308 11.41  26.95 85.06   57.47
## 4 V1JoRpA7sT7eViQ56cG4Iw_shi.txt       1 397  7.35  32.24 79.35   42.07
## 5   f_WfzmORHwvQs8-6Zt6cdQ_d.txt       1 273 13.65  37.73 81.68   43.96
## 6   ETG-F33qQmGXZ8IYjhLmWw_d.txt       1 154  9.06  22.08 87.01   53.90
##   Местоимение Личноеместоимение     я   мы   ты онаон  они
## 1        9.38              9.38  3.12 4.69   NA  0.78 0.78
## 2       15.49              7.75  0.70 4.93   NA  1.41 4.93
## 3       16.88             13.31  9.42 0.32 0.32  3.25 2.27
## 4       15.37             11.34  9.07   NA   NA  1.26 2.77
## 5       11.36              9.16  5.13 1.47   NA  2.56 1.10
## 6       19.48             16.23 11.04 0.65   NA  4.55 3.25
##   неопределенныеместоимен вы вербальные сленг Наречие Предлог  Союз Отрицание
## 1                      NA NA      11.72    NA    1.56   12.50 10.16      1.56
## 2                    0.70 NA      10.56    NA      NA   11.27  5.63      2.11
## 3                    0.65 NA      18.83    NA    4.55   14.94 11.69      4.22
## 4                    1.26 NA      17.13    NA    2.77    7.56  8.56      2.77
## 5                    0.37 NA      14.29    NA    2.56    8.42 13.55      2.20
## 6                      NA NA      17.53    NA    4.55   11.69  9.09      0.65
##   quant Числительное ВульгаризмТабу Общество Семья Друг Человечество Действие
## 1  7.81         1.56             NA    15.62    NA   NA         6.25     7.81
## 2  6.34           NA             NA     8.45    NA   NA           NA    10.56
## 3  4.55           NA             NA     8.77  0.97 0.32           NA     8.12
## 4  2.52         0.50             NA     7.81    NA   NA           NA    19.65
## 5  4.76         1.10             NA    10.26    NA   NA         4.40     9.52
## 6  2.60         0.65             NA    11.04  1.30   NA         3.25    13.64
##   Позитив Негатив Беспокойство Гнев Грусть Когнитив Интуиция Мотивация
## 1    5.47    1.56           NA   NA     NA    26.56     1.56      2.34
## 2    7.75    2.11           NA 1.41     NA    19.01     5.63      0.70
## 3    3.25    4.87         0.65 1.95   0.65    21.43     1.95      3.90
## 4   14.61    3.78         3.53 1.01   0.25    23.17     4.53      2.27
## 5    8.06    0.37           NA 0.37     NA    23.44     2.93      4.03
## 6   11.69    1.95           NA   NA   1.30    19.48     4.55      1.30
##   Несоответствие Попытка Уверенность Помеха Включение Исключение Перцепция
## 1           1.56    2.34        4.69   0.78     13.28       0.78      3.12
## 2             NA    0.70        4.23   0.70      9.15         NA      0.70
## 3           0.97    3.25        3.25   0.32      8.77       1.62      0.97
## 4           2.77    2.52        5.04   1.76      5.04       4.03      1.76
## 5           1.47    2.56        3.30   1.10     10.99       2.20      2.20
## 6           0.65    1.30        1.95   1.30     11.69       0.65      1.95
##   Видение Слышание Чувствование Биология Тело Здоровье Секс Потребление
## 1    1.56     0.78           NA     3.12   NA     2.34 0.78          NA
## 2    0.70       NA           NA     3.52   NA     2.82   NA        0.70
## 3    0.32     0.32         0.32     0.97 0.32     0.65   NA          NA
## 4    0.76     0.76         0.25     3.78 0.50     1.26 1.26        1.01
## 5    1.47       NA         0.73     2.93   NA     1.83   NA        1.10
## 6    0.65       NA         1.30     3.25   NA     1.95 1.30          NA
##   Сравнение Стимул Пространство Время Работа Успех Досуг  Дом Деньги Религия
## 1     28.12   3.91        15.62  6.25   0.78  1.56    NA   NA     NA      NA
## 2     23.94   3.52        16.90  0.70   0.70  2.82    NA   NA   1.41      NA
## 3     21.10   1.30        13.96  4.55   3.57  2.60  0.65   NA     NA    0.65
## 4     15.11   2.77         8.31  4.53   0.25  4.79    NA   NA   0.25      NA
## 5     15.38   2.20        10.99  2.20   3.66  2.56  0.37 0.73   1.47      NA
## 6     29.87   2.60        18.18  7.79   0.65  0.65  1.95 0.65     NA      NA
##   Смерть Согласие Междометие Наполнение AllPunc Period Comma Colon SemiC QMark
## 1     NA       NA         NA       0.78   24.22  10.16 11.72    NA    NA    NA
## 2    0.7      0.7         NA         NA   19.72   7.04  9.86    NA    NA    NA
## 3     NA      1.3         NA       0.97   20.13   9.42  8.77  0.32    NA    NA
## 4     NA       NA         NA       0.76   34.76  12.59 10.58    NA   0.5  1.01
## 5     NA       NA         NA       2.20   16.48   7.33  5.86  1.47    NA    NA
## 6     NA       NA         NA       0.65   24.03  11.04  6.49  0.65    NA    NA
##   Exclam Dash Quote Apostro Parenth OtherP  group
## 1     NA 0.78  1.56      NA      NA     NA clinic
## 2     NA 1.41  1.41      NA      NA     NA clinic
## 3     NA 0.65  0.65      NA      NA   0.32 clinic
## 4   0.76 0.50  8.56      NA      NA   0.25 clinic
## 5     NA 0.37  0.73      NA    0.73     NA clinic
## 6     NA 0.65  5.19      NA      NA     NA clinic
\end{verbatim}

This is concatenation of LIWC analysis using LIWC2015 + default russian
dictionary. There is a lot of columns representing different word
categories, but we only need those related to emotiona expression.

\begin{Shaded}
\begin{Highlighting}[]
\NormalTok{needed\_columns }\OtherTok{\textless{}{-}} \FunctionTok{c}\NormalTok{(}\StringTok{"Filename"}\NormalTok{, }\StringTok{"group"}\NormalTok{, }\StringTok{"Позитив"}\NormalTok{, }\StringTok{"Негатив"}\NormalTok{, }\StringTok{"Беспокойство"}\NormalTok{, }\StringTok{"Гнев"}\NormalTok{, }\StringTok{"Грусть"}\NormalTok{)}
\NormalTok{data }\OtherTok{\textless{}{-}}\NormalTok{ data[, needed\_columns]}
\FunctionTok{head}\NormalTok{(data)}
\end{Highlighting}
\end{Shaded}

\begin{verbatim}
##                         Filename  group Позитив Негатив Беспокойство Гнев
## 1   E2T12HF7DVgDyo2apqO3Qw_d.txt clinic    5.47    1.56           NA   NA
## 2   E6HnYOg_LBTxiFKNQbsNSw_d.txt clinic    7.75    2.11           NA 1.41
## 3   Y-oL9e5ZdMN5DytW7XMqGg_d.txt clinic    3.25    4.87         0.65 1.95
## 4 V1JoRpA7sT7eViQ56cG4Iw_shi.txt clinic   14.61    3.78         3.53 1.01
## 5   f_WfzmORHwvQs8-6Zt6cdQ_d.txt clinic    8.06    0.37           NA 0.37
## 6   ETG-F33qQmGXZ8IYjhLmWw_d.txt clinic   11.69    1.95           NA   NA
##   Грусть
## 1     NA
## 2     NA
## 3   0.65
## 4   0.25
## 5     NA
## 6   1.30
\end{verbatim}

Also, let's rename the columns for convenience.

\begin{Shaded}
\begin{Highlighting}[]
\NormalTok{columns }\OtherTok{\textless{}{-}} \FunctionTok{c}\NormalTok{(}\StringTok{"file"}\NormalTok{, }\StringTok{"group"}\NormalTok{, }\StringTok{"Positive"}\NormalTok{, }\StringTok{"Negative"}\NormalTok{, }\StringTok{"Anxienty"}\NormalTok{, }\StringTok{"Anger"}\NormalTok{, }\StringTok{"Sadness"}\NormalTok{)}
\NormalTok{emotion\_columns }\OtherTok{\textless{}{-}} \FunctionTok{c}\NormalTok{(}\StringTok{"Positive"}\NormalTok{, }\StringTok{"Negative"}\NormalTok{, }\StringTok{"Anxienty"}\NormalTok{, }\StringTok{"Anger"}\NormalTok{, }\StringTok{"Sadness"}\NormalTok{)}
\FunctionTok{colnames}\NormalTok{(data) }\OtherTok{\textless{}{-}}\NormalTok{ columns}
\end{Highlighting}
\end{Shaded}

Next, we will perform some exploratory analysis, checking what we are
working on.

\begin{Shaded}
\begin{Highlighting}[]
\FunctionTok{group\_by}\NormalTok{(data, group) }\SpecialCharTok{\%\textgreater{}\%}
  \FunctionTok{summarise}\NormalTok{(}
    \AttributeTok{count =} \FunctionTok{n}\NormalTok{(),}
    \FunctionTok{across}\NormalTok{(}
\NormalTok{      emotion\_columns,}
      \FunctionTok{list}\NormalTok{(}
        \AttributeTok{mean =} \SpecialCharTok{\textasciitilde{}} \FunctionTok{mean}\NormalTok{(.x, }\AttributeTok{na.rm =} \ConstantTok{TRUE}\NormalTok{),}
        \AttributeTok{sd =} \SpecialCharTok{\textasciitilde{}} \FunctionTok{sd}\NormalTok{(.x, }\AttributeTok{na.rm =} \ConstantTok{TRUE}\NormalTok{),}
        \AttributeTok{median =} \SpecialCharTok{\textasciitilde{}} \FunctionTok{median}\NormalTok{(.x, }\AttributeTok{na.rm =} \ConstantTok{TRUE}\NormalTok{)}
\NormalTok{      )}
\NormalTok{    )}
\NormalTok{  )}
\end{Highlighting}
\end{Shaded}

\begin{verbatim}
## Note: Using an external vector in selections is ambiguous.
## i Use `all_of(emotion_columns)` instead of `emotion_columns` to silence this message.
## i See <https://tidyselect.r-lib.org/reference/faq-external-vector.html>.
## This message is displayed once per session.
\end{verbatim}

\begin{verbatim}
## # A tibble: 5 x 17
##   group   count Positive_mean Positive_sd Positive_median Negative_mean
##   <chr>   <int>         <dbl>       <dbl>           <dbl>         <dbl>
## 1 clinic     18          8.94        4.76            7.90          2.58
## 2 control    22          7.04        2.45            6.71          1.67
## 3 psy       111          7.61        2.43            7.80          1.95
## 4 soc        26          8.55        2.71            8.55          2.21
## 5 stem       13          7.32        1.91            7.36          2.33
## # ... with 11 more variables: Negative_sd <dbl>, Negative_median <dbl>,
## #   Anxienty_mean <dbl>, Anxienty_sd <dbl>, Anxienty_median <dbl>,
## #   Anger_mean <dbl>, Anger_sd <dbl>, Anger_median <dbl>, Sadness_mean <dbl>,
## #   Sadness_sd <dbl>, Sadness_median <dbl>
\end{verbatim}

So, in general values have high SD + due to liwc limitations. Let's
display the same information in a visual way.

\begin{Shaded}
\begin{Highlighting}[]
\ControlFlowTok{for}\NormalTok{(col }\ControlFlowTok{in}\NormalTok{ emotion\_columns) \{}
  \FunctionTok{ggboxplot}\NormalTok{(}
\NormalTok{    data, }
    \AttributeTok{x =} \StringTok{"group"}\NormalTok{, }\AttributeTok{y =}\NormalTok{ col, }
    \AttributeTok{color =} \StringTok{"group"}\NormalTok{,}
    \AttributeTok{ylab =}\NormalTok{ col, }\AttributeTok{xlab =} \StringTok{"Group"}
\NormalTok{  )}
  \FunctionTok{ggsave}\NormalTok{(}
    \FunctionTok{paste}\NormalTok{(}\StringTok{"plots/boxplots/"}\NormalTok{, col, }\StringTok{".pdf"}\NormalTok{, }\AttributeTok{sep=}\StringTok{\textquotesingle{}\textquotesingle{}}\NormalTok{)}
\NormalTok{  )}
\NormalTok{\}}
\end{Highlighting}
\end{Shaded}

\begin{verbatim}
## Saving 6.5 x 4.5 in image
\end{verbatim}

\begin{verbatim}
## Warning: Removed 33 rows containing non-finite values (stat_boxplot).
\end{verbatim}

\begin{verbatim}
## Saving 6.5 x 4.5 in image
\end{verbatim}

\begin{verbatim}
## Warning: Removed 37 rows containing non-finite values (stat_boxplot).
\end{verbatim}

\begin{verbatim}
## Saving 6.5 x 4.5 in image
\end{verbatim}

\begin{verbatim}
## Warning: Removed 63 rows containing non-finite values (stat_boxplot).
\end{verbatim}

\begin{verbatim}
## Saving 6.5 x 4.5 in image
\end{verbatim}

\begin{verbatim}
## Warning: Removed 78 rows containing non-finite values (stat_boxplot).
\end{verbatim}

\begin{verbatim}
## Saving 6.5 x 4.5 in image
\end{verbatim}

\begin{verbatim}
## Warning: Removed 78 rows containing non-finite values (stat_boxplot).
\end{verbatim}

We need to examine the distribution of values overall and inside each
group. Let's do it by looking at histogram, Q-Q plot + performing
Shapiro Wilk's test.

\begin{Shaded}
\begin{Highlighting}[]
\ControlFlowTok{for}\NormalTok{(col }\ControlFlowTok{in}\NormalTok{ emotion\_columns) \{}
  \FunctionTok{ggqqplot}\NormalTok{(}
    \FunctionTok{get}\NormalTok{(col, data),}
    \AttributeTok{title=}\FunctionTok{paste}\NormalTok{(}\StringTok{"Q{-}Q plot for "}\NormalTok{, }\StringTok{"Positive"}\NormalTok{, }\AttributeTok{sep=}\StringTok{\textquotesingle{}\textquotesingle{}}\NormalTok{)}
\NormalTok{  )}
  \FunctionTok{ggsave}\NormalTok{(}
    \FunctionTok{paste}\NormalTok{(}\StringTok{"plots/qqplots/"}\NormalTok{, col, }\StringTok{".pdf"}\NormalTok{, }\AttributeTok{sep=}\StringTok{\textquotesingle{}\textquotesingle{}}\NormalTok{)}
\NormalTok{  )}
\NormalTok{\}}
\end{Highlighting}
\end{Shaded}

\begin{verbatim}
## Saving 6.5 x 4.5 in image
\end{verbatim}

\begin{verbatim}
## Warning: Removed 33 rows containing non-finite values (stat_qq).
\end{verbatim}

\begin{verbatim}
## Warning: Removed 33 rows containing non-finite values (stat_qq_line).

## Warning: Removed 33 rows containing non-finite values (stat_qq_line).
\end{verbatim}

\begin{verbatim}
## Saving 6.5 x 4.5 in image
\end{verbatim}

\begin{verbatim}
## Warning: Removed 37 rows containing non-finite values (stat_qq).
\end{verbatim}

\begin{verbatim}
## Warning: Removed 37 rows containing non-finite values (stat_qq_line).

## Warning: Removed 37 rows containing non-finite values (stat_qq_line).
\end{verbatim}

\begin{verbatim}
## Saving 6.5 x 4.5 in image
\end{verbatim}

\begin{verbatim}
## Warning: Removed 63 rows containing non-finite values (stat_qq).
\end{verbatim}

\begin{verbatim}
## Warning: Removed 63 rows containing non-finite values (stat_qq_line).

## Warning: Removed 63 rows containing non-finite values (stat_qq_line).
\end{verbatim}

\begin{verbatim}
## Saving 6.5 x 4.5 in image
\end{verbatim}

\begin{verbatim}
## Warning: Removed 78 rows containing non-finite values (stat_qq).
\end{verbatim}

\begin{verbatim}
## Warning: Removed 78 rows containing non-finite values (stat_qq_line).

## Warning: Removed 78 rows containing non-finite values (stat_qq_line).
\end{verbatim}

\begin{verbatim}
## Saving 6.5 x 4.5 in image
\end{verbatim}

\begin{verbatim}
## Warning: Removed 78 rows containing non-finite values (stat_qq).

## Warning: Removed 78 rows containing non-finite values (stat_qq_line).

## Warning: Removed 78 rows containing non-finite values (stat_qq_line).
\end{verbatim}

Our data partially fits Q-Q plot theoretical part, so it may be normal
distribution. But there is also some deviations. Guess we need to look
at histograms.

\begin{Shaded}
\begin{Highlighting}[]
\ControlFlowTok{for}\NormalTok{(col }\ControlFlowTok{in}\NormalTok{ emotion\_columns) \{}
  \FunctionTok{ggplot}\NormalTok{(data, }\FunctionTok{aes\_string}\NormalTok{(}\AttributeTok{x=}\NormalTok{col)) }\SpecialCharTok{+}
    \FunctionTok{geom\_histogram}\NormalTok{(}
      \FunctionTok{aes}\NormalTok{(}\AttributeTok{y=}\NormalTok{..density..),}
      \AttributeTok{position=}\StringTok{"identity"}\NormalTok{,}
      \AttributeTok{colour=}\StringTok{"black"}\NormalTok{, }\AttributeTok{fill=}\StringTok{"white"}\NormalTok{,}
      \AttributeTok{alpha=}\FloatTok{0.5}
\NormalTok{    ) }\SpecialCharTok{+}
    \FunctionTok{geom\_density}\NormalTok{(}\AttributeTok{alpha=}\FloatTok{0.2}\NormalTok{, }\AttributeTok{fill=}\StringTok{"\#FF6666"}\NormalTok{)}
  \FunctionTok{ggsave}\NormalTok{(}
    \FunctionTok{paste}\NormalTok{(}\StringTok{"plots/histogram/"}\NormalTok{, col, }\StringTok{".pdf"}\NormalTok{, }\AttributeTok{sep=}\StringTok{""}\NormalTok{)}
\NormalTok{  )}
\NormalTok{\}}
\end{Highlighting}
\end{Shaded}

\begin{verbatim}
## Saving 6.5 x 4.5 in image
\end{verbatim}

\begin{verbatim}
## `stat_bin()` using `bins = 30`. Pick better value with `binwidth`.
\end{verbatim}

\begin{verbatim}
## Warning: Removed 33 rows containing non-finite values (stat_bin).
\end{verbatim}

\begin{verbatim}
## Warning: Removed 33 rows containing non-finite values (stat_density).
\end{verbatim}

\begin{verbatim}
## Saving 6.5 x 4.5 in image
## `stat_bin()` using `bins = 30`. Pick better value with `binwidth`.
\end{verbatim}

\begin{verbatim}
## Warning: Removed 37 rows containing non-finite values (stat_bin).
\end{verbatim}

\begin{verbatim}
## Warning: Removed 37 rows containing non-finite values (stat_density).
\end{verbatim}

\begin{verbatim}
## Saving 6.5 x 4.5 in image
## `stat_bin()` using `bins = 30`. Pick better value with `binwidth`.
\end{verbatim}

\begin{verbatim}
## Warning: Removed 63 rows containing non-finite values (stat_bin).
\end{verbatim}

\begin{verbatim}
## Warning: Removed 63 rows containing non-finite values (stat_density).
\end{verbatim}

\begin{verbatim}
## Saving 6.5 x 4.5 in image
## `stat_bin()` using `bins = 30`. Pick better value with `binwidth`.
\end{verbatim}

\begin{verbatim}
## Warning: Removed 78 rows containing non-finite values (stat_bin).
\end{verbatim}

\begin{verbatim}
## Warning: Removed 78 rows containing non-finite values (stat_density).
\end{verbatim}

\begin{verbatim}
## Saving 6.5 x 4.5 in image
## `stat_bin()` using `bins = 30`. Pick better value with `binwidth`.
\end{verbatim}

\begin{verbatim}
## Warning: Removed 78 rows containing non-finite values (stat_bin).

## Warning: Removed 78 rows containing non-finite values (stat_density).
\end{verbatim}

By looking at histograms, we can say that some of our categories are
more likely to be from exponential distribution(Anger, Anxiety, Sadness)
and some are more likely ot be from Normal distribution(Positive,
Negative). Probably it is save to say that our data generally do not
come from normal distribtuion, but additionally, we will perform
Shapiro-Wilk test to check of any values are \texttt{normal} enough. We
will use \(\alpha = 0.001\) as our significance level.

\begin{Shaded}
\begin{Highlighting}[]
\ControlFlowTok{for}\NormalTok{(col }\ControlFlowTok{in}\NormalTok{ emotion\_columns) \{}
  \FunctionTok{print}\NormalTok{(}\StringTok{"============================="}\NormalTok{)}
  \FunctionTok{print}\NormalTok{(col)}
  \FunctionTok{print}\NormalTok{(}\FunctionTok{shapiro.test}\NormalTok{(}\FunctionTok{get}\NormalTok{(col, data)))}
\NormalTok{\}}
\end{Highlighting}
\end{Shaded}

\begin{verbatim}
## [1] "============================="
## [1] "Positive"
## 
##  Shapiro-Wilk normality test
## 
## data:  get(col, data)
## W = 0.93893, p-value = 2.747e-06
## 
## [1] "============================="
## [1] "Negative"
## 
##  Shapiro-Wilk normality test
## 
## data:  get(col, data)
## W = 0.92944, p-value = 7.303e-07
## 
## [1] "============================="
## [1] "Anxienty"
## 
##  Shapiro-Wilk normality test
## 
## data:  get(col, data)
## W = 0.8027, p-value = 8.903e-12
## 
## [1] "============================="
## [1] "Anger"
## 
##  Shapiro-Wilk normality test
## 
## data:  get(col, data)
## W = 0.81854, p-value = 2.001e-10
## 
## [1] "============================="
## [1] "Sadness"
## 
##  Shapiro-Wilk normality test
## 
## data:  get(col, data)
## W = 0.78618, p-value = 1.793e-11
\end{verbatim}

As none of the tests returned p-value \textgreater{} \(\alpha = 0.001\),
we do reject Null hypothesis(which states that our data comes from
normal distribution).

\hypertarget{lets-move-into-actual-hypothesis-testing}{%
\section{Let's move into actual hypothesis
testing}\label{lets-move-into-actual-hypothesis-testing}}

Mainly we want to learn if any of groups of people used emotional words
differently(more or less frequently) than the control group. And if this
is true, we want to know the groups and confidence interval of medians
of values.

Firstly, we will check if the effect(using emotional wording differently
among groups) is present at all. For this, the Kruskal-Wallis test to be
used. For this test the null hypothesis is that parameter distribution
is the same for every group. We will use significance level of
\(\alpha = 0.1\).

\begin{Shaded}
\begin{Highlighting}[]
\ControlFlowTok{for}\NormalTok{(col }\ControlFlowTok{in}\NormalTok{ emotion\_columns) \{}
\NormalTok{  formula }\OtherTok{\textless{}{-}} \FunctionTok{paste}\NormalTok{(col, }\StringTok{"\textasciitilde{} group"}\NormalTok{)}
  \FunctionTok{print}\NormalTok{(}\StringTok{"====================================================="}\NormalTok{)}
  \FunctionTok{print}\NormalTok{(formula)}
  \FunctionTok{print}\NormalTok{(}\FunctionTok{kruskal.test}\NormalTok{(}\FunctionTok{as.formula}\NormalTok{(formula), }\AttributeTok{data =}\NormalTok{ data, }\AttributeTok{na.action =}\NormalTok{ na.exclude))}
\NormalTok{\}}
\end{Highlighting}
\end{Shaded}

\begin{verbatim}
## [1] "====================================================="
## [1] "Positive ~ group"
## 
##  Kruskal-Wallis rank sum test
## 
## data:  Positive by group
## Kruskal-Wallis chi-squared = 5.0461, df = 4, p-value = 0.2826
## 
## [1] "====================================================="
## [1] "Negative ~ group"
## 
##  Kruskal-Wallis rank sum test
## 
## data:  Negative by group
## Kruskal-Wallis chi-squared = 5.5818, df = 4, p-value = 0.2326
## 
## [1] "====================================================="
## [1] "Anxienty ~ group"
## 
##  Kruskal-Wallis rank sum test
## 
## data:  Anxienty by group
## Kruskal-Wallis chi-squared = 6.7837, df = 4, p-value = 0.1478
## 
## [1] "====================================================="
## [1] "Anger ~ group"
## 
##  Kruskal-Wallis rank sum test
## 
## data:  Anger by group
## Kruskal-Wallis chi-squared = 10.631, df = 4, p-value = 0.03104
## 
## [1] "====================================================="
## [1] "Sadness ~ group"
## 
##  Kruskal-Wallis rank sum test
## 
## data:  Sadness by group
## Kruskal-Wallis chi-squared = 5.0891, df = 4, p-value = 0.2783
\end{verbatim}

Only for Anger word category the kruskal test p-value is less than
significance level. That means that in our data the effect is present
only for Anger. Let's apply pairwise Wilcoxon(Mann-Whitney) test to see
which groups are different. Significance level remains the same:
\(\alpha = 0.1\).

\begin{Shaded}
\begin{Highlighting}[]
\NormalTok{data }\SpecialCharTok{\%\textgreater{}\%} \FunctionTok{pairwise\_wilcox\_test}\NormalTok{(}
\NormalTok{  Anger }\SpecialCharTok{\textasciitilde{}}\NormalTok{ group,}
  \AttributeTok{p.adjust.method =} \StringTok{"BH"}\NormalTok{,}
  \AttributeTok{ref.group =} \StringTok{"control"}
\NormalTok{)}
\end{Highlighting}
\end{Shaded}

\begin{verbatim}
## # A tibble: 4 x 9
##   .y.   group1  group2    n1    n2 statistic     p p.adj p.adj.signif
## * <chr> <chr>   <chr>  <int> <int>     <dbl> <dbl> <dbl> <chr>       
## 1 Anger control clinic    22    18      34.5 0.009 0.036 *           
## 2 Anger control psy       22   111     393   0.46  0.613 ns          
## 3 Anger control soc       22    26      90   0.027 0.054 ns          
## 4 Anger control stem      22    13      69.5 0.91  0.91  ns
\end{verbatim}

As we see, we can reject null-hypothesis for Mann-Whitney test for
following pairs: - Clinic / Control - Soc / Control

Let's estimate effect\_size:

\begin{Shaded}
\begin{Highlighting}[]
\NormalTok{control\_clinic }\OtherTok{=}\NormalTok{ data }\SpecialCharTok{\%\textgreater{}\%} \FunctionTok{filter}\NormalTok{(group }\SpecialCharTok{==} \StringTok{"control"} \SpecialCharTok{|}\NormalTok{ group }\SpecialCharTok{==} \StringTok{"clinic"}\NormalTok{)}
\NormalTok{control\_soc }\OtherTok{=}\NormalTok{ data }\SpecialCharTok{\%\textgreater{}\%} \FunctionTok{filter}\NormalTok{(group }\SpecialCharTok{==} \StringTok{"control"} \SpecialCharTok{|}\NormalTok{ group }\SpecialCharTok{==} \StringTok{"soc"}\NormalTok{)}
\NormalTok{control\_psy }\OtherTok{=}\NormalTok{ data }\SpecialCharTok{\%\textgreater{}\%} \FunctionTok{filter}\NormalTok{(group }\SpecialCharTok{==} \StringTok{"control"} \SpecialCharTok{|}\NormalTok{ group }\SpecialCharTok{==} \StringTok{"psy"}\NormalTok{)}
\NormalTok{control\_stem }\OtherTok{=}\NormalTok{ data }\SpecialCharTok{\%\textgreater{}\%} \FunctionTok{filter}\NormalTok{(group }\SpecialCharTok{==} \StringTok{"control"} \SpecialCharTok{|}\NormalTok{ group }\SpecialCharTok{==} \StringTok{"stem"}\NormalTok{)}
\FunctionTok{print}\NormalTok{(}\FunctionTok{wilcox\_effsize}\NormalTok{(control\_clinic, Anger }\SpecialCharTok{\textasciitilde{}}\NormalTok{ group))}
\end{Highlighting}
\end{Shaded}

\begin{verbatim}
## # A tibble: 1 x 7
##   .y.   group1 group2  effsize    n1    n2 magnitude
## * <chr> <chr>  <chr>     <dbl> <int> <int> <ord>    
## 1 Anger clinic control   0.508    18    22 large
\end{verbatim}

\begin{Shaded}
\begin{Highlighting}[]
\FunctionTok{print}\NormalTok{(}\FunctionTok{wilcox\_effsize}\NormalTok{(control\_soc, Anger }\SpecialCharTok{\textasciitilde{}}\NormalTok{ group))}
\end{Highlighting}
\end{Shaded}

\begin{verbatim}
## # A tibble: 1 x 7
##   .y.   group1  group2 effsize    n1    n2 magnitude
## * <chr> <chr>   <chr>    <dbl> <int> <int> <ord>    
## 1 Anger control soc      0.371    22    26 moderate
\end{verbatim}

\begin{Shaded}
\begin{Highlighting}[]
\FunctionTok{print}\NormalTok{(}\FunctionTok{wilcox\_effsize}\NormalTok{(control\_psy, Anger }\SpecialCharTok{\textasciitilde{}}\NormalTok{ group))}
\end{Highlighting}
\end{Shaded}

\begin{verbatim}
## # A tibble: 1 x 7
##   .y.   group1  group2 effsize    n1    n2 magnitude
## * <chr> <chr>   <chr>    <dbl> <int> <int> <ord>    
## 1 Anger control psy     0.0878    22   111 small
\end{verbatim}

\begin{Shaded}
\begin{Highlighting}[]
\FunctionTok{print}\NormalTok{(}\FunctionTok{wilcox\_effsize}\NormalTok{(control\_stem, Anger }\SpecialCharTok{\textasciitilde{}}\NormalTok{ group))}
\end{Highlighting}
\end{Shaded}

\begin{verbatim}
## # A tibble: 1 x 7
##   .y.   group1  group2 effsize    n1    n2 magnitude
## * <chr> <chr>   <chr>    <dbl> <int> <int> <ord>    
## 1 Anger control stem    0.0283    22    13 small
\end{verbatim}

Finally, we will compute CI of medians in each group:

\begin{Shaded}
\begin{Highlighting}[]
\NormalTok{control }\OtherTok{=}\NormalTok{ data }\SpecialCharTok{\%\textgreater{}\%} \FunctionTok{filter}\NormalTok{(group }\SpecialCharTok{==} \StringTok{"control"}\NormalTok{)}
\NormalTok{clinic }\OtherTok{=}\NormalTok{ data }\SpecialCharTok{\%\textgreater{}\%} \FunctionTok{filter}\NormalTok{(group }\SpecialCharTok{==} \StringTok{"clinic"}\NormalTok{)}
\NormalTok{soc }\OtherTok{=}\NormalTok{ data }\SpecialCharTok{\%\textgreater{}\%} \FunctionTok{filter}\NormalTok{(group }\SpecialCharTok{==} \StringTok{"soc"}\NormalTok{)}
\FunctionTok{print}\NormalTok{(}\StringTok{"Control"}\NormalTok{)}
\end{Highlighting}
\end{Shaded}

\begin{verbatim}
## [1] "Control"
\end{verbatim}

\begin{Shaded}
\begin{Highlighting}[]
\FunctionTok{print}\NormalTok{(}\FunctionTok{wilcox.test}\NormalTok{(control}\SpecialCharTok{$}\NormalTok{Anger, }\AttributeTok{conf.int =} \ConstantTok{TRUE}\NormalTok{, }\AttributeTok{conf.level =} \FloatTok{0.95}\NormalTok{)}\SpecialCharTok{$}\NormalTok{conf.int)}
\end{Highlighting}
\end{Shaded}

\begin{verbatim}
## Warning in wilcox.test.default(control$Anger, conf.int = TRUE, conf.level =
## 0.95): cannot compute exact p-value with ties
\end{verbatim}

\begin{verbatim}
## Warning in wilcox.test.default(control$Anger, conf.int = TRUE, conf.level =
## 0.95): cannot compute exact confidence interval with ties
\end{verbatim}

\begin{verbatim}
## [1] 0.3850466 0.8299455
## attr(,"conf.level")
## [1] 0.95
\end{verbatim}

\begin{Shaded}
\begin{Highlighting}[]
\FunctionTok{print}\NormalTok{(}\StringTok{"Clinic"}\NormalTok{)}
\end{Highlighting}
\end{Shaded}

\begin{verbatim}
## [1] "Clinic"
\end{verbatim}

\begin{Shaded}
\begin{Highlighting}[]
\FunctionTok{print}\NormalTok{(}\FunctionTok{wilcox.test}\NormalTok{(clinic}\SpecialCharTok{$}\NormalTok{Anger, }\AttributeTok{conf.int =} \ConstantTok{TRUE}\NormalTok{, }\AttributeTok{conf.level =} \FloatTok{0.95}\NormalTok{)}\SpecialCharTok{$}\NormalTok{conf.int)}
\end{Highlighting}
\end{Shaded}

\begin{verbatim}
## [1] 0.715 1.935
## attr(,"conf.level")
## [1] 0.95
\end{verbatim}

\begin{Shaded}
\begin{Highlighting}[]
\FunctionTok{print}\NormalTok{(}\StringTok{"Soc"}\NormalTok{)}
\end{Highlighting}
\end{Shaded}

\begin{verbatim}
## [1] "Soc"
\end{verbatim}

\begin{Shaded}
\begin{Highlighting}[]
\FunctionTok{print}\NormalTok{(}\FunctionTok{wilcox.test}\NormalTok{(soc}\SpecialCharTok{$}\NormalTok{Anger, }\AttributeTok{conf.int =} \ConstantTok{TRUE}\NormalTok{, }\AttributeTok{conf.level =} \FloatTok{0.95}\NormalTok{)}\SpecialCharTok{$}\NormalTok{conf.int)}
\end{Highlighting}
\end{Shaded}

\begin{verbatim}
## Warning in wilcox.test.default(soc$Anger, conf.int = TRUE, conf.level = 0.95):
## cannot compute exact p-value with ties
\end{verbatim}

\begin{verbatim}
## Warning in wilcox.test.default(soc$Anger, conf.int = TRUE, conf.level = 0.95):
## cannot compute exact confidence interval with ties
\end{verbatim}

\begin{verbatim}
## [1] 0.6949834 1.4450645
## attr(,"conf.level")
## [1] 0.95
\end{verbatim}

\end{document}
